% !TEX root =  main.tex

In this section, we give details of computing the Casimir energy via boundary integral operator discretisations. 
Assume 
$\Omega^{-}\in\mathbb{R}^{d}$, for $d \geq 2$ is the interior bounded Lebesgue-measurable domain that the scatterer occupies with piecewise smooth Lipschitz boundary $\Gamma$. The exterior domain is denoted as 
$\Omega^{+} = \mathbb{R}^{d}\backslash\overline{\Omega^{-}}$. $\boldsymbol{n}$ is the exterior unit normal to the surface $\Gamma$ pointing outwards from $\Omega^{-}$ and 
$\boldsymbol{n}_{\boldsymbol{x}}$ is normal to $\Gamma$ at the point $\boldsymbol{x}\in\Gamma$.

In the scalar case, the Casimir energy can be expressed in terms of certain single-layer boundary operator, which we will define below. We then present its relationship with the Krein-Spectral shift function and demonstrate how it can practically be computed.

\subsection{The single-layer boundary operator}
For the bounded interior domain $\Omega^{-}$ or the unbounded exterior domain $\Omega^{+}$, the space of the (locally) square integrable functions is 
\begin{align*}
    L^{2}(\Omega^{-}) &:= \left\{f:\Omega^{\pm}\rightarrow\mathbb{C}, f \text{ is Lebesgue measurable and} \int_{\Omega^{-}}|f|^{2} < \infty \right\},\\
    L_{\text{loc}}^{2}(\Omega^{+}) &:= \left\{f:\Omega^{+}\rightarrow\mathbb{C},\ f \text{ is Lebesgue measurable and} \int_{K}|f|^{2} < \infty, \ \text{for all compact}\ K \subset \Omega^{+} \right\}
\end{align*}
and note that the subscript ``loc'' can be removed if the domain is bounded (i.e. $L_{\text{loc}}^{2}(\Omega^{-}) = L^{2}(\Omega^{-})$).
We define the (local) Sobolev space $H_{\text{loc}}^{s}(\Omega^{\pm})$ as 
\begin{align*}
    H_{\text{loc}}^{s}(\Omega^{\pm}):=\left\{f\in L_{\text{loc}}^{2}(\Omega^{\pm}), \forall\alpha \text{ s.t.} |\alpha|\leq s, D^{\alpha}f\in L_{\text{loc}}^{2}(\Omega^{\pm})\right\},
\end{align*}
where $\alpha = (\alpha_{1}, \alpha_{2}, \dots, \alpha_{d})$ is a multi-index and $|\alpha| = \alpha_{1} + \alpha_{2} + \dots + \alpha_{d}$, and 
the derivative is defined in the weak sense.

For any function $f\in H_{\text{loc}}^1(\Omega)$ we can define the trace $\gamma_{D}^{\pm}$ as
\begin{align*}
    \gamma_{\text{D}}^{\pm}p(\boldsymbol{x}):=\lim_{\Omega^{\pm}\ni\boldsymbol{x'}\rightarrow\boldsymbol{x}\in\Gamma}p(\boldsymbol{x'}).
\end{align*}
We call the range of the trace operator $H^{1/2}(\Gamma)$. This space can be more rigorously defined. But for the purposes of this paper the description of $H^{1/2}(\Gamma)$ is range space of the trace operator is sufficient. We furthermore need the space $H^{-1/2}(\Gamma)$, which is the dual space of $H^{1/2}(\Gamma)$ with $L^2(\Gamma)$ as pivot space.

We can now define the single-layer boundary $\mathcal{V}:H^{-1/2}(\Gamma)\rightarrow H^{1/2}(\Gamma)$ as

\begin{align*}
    (V_{k}\mu)(\boldsymbol{x}) := \int_{\Gamma}g_{k}(\boldsymbol{x},\boldsymbol{y})\psi(\boldsymbol{y})dS_{\boldsymbol{y}}, \ \ \ \ \ 
    \text{for}\ \mu\in H^{-\frac{1}{2}}(\Gamma) \  \text{and} \ \boldsymbol{x}\in\Gamma.
\end{align*}
Here, 
\begin{align}\label{Green's function}
    g_{k}(\boldsymbol{x},\boldsymbol{y}) = \begin{cases}
          \frac{\mathrm{i}}{4}H_{0}^{(1)}(k|\boldsymbol{x}-\boldsymbol{y}|), \ \ \ \ &\text{for} \ d = 2\\
          \frac{e^{ik|\boldsymbol{x}-\boldsymbol{y}|}}{4\pi|\boldsymbol{x} - \boldsymbol{y}|}, \ \ \ \ &\text{for} \ d = 3,
        \end{cases}
\end{align}
with $H_{0}^{(1)}$  a Hankel function of the first kind.



\subsection{Relation between the Krein spectral shift function and the single-layer boundary operator}
By \cite{hanisch2020relative}, the Krein spectral shift function is defined as 
\begin{align*}
    \xi(k) = \frac{1}{2\pi \mathrm{i}}\log\left(\frac{\det(S(k))}{\det(S_{1,k})\cdots\det(S_{N,k})}\right),
\end{align*}
where $S_{i,n}$ is the scattering matrix associated with the $n$th scatterer. These scattering matrices can be constructed  $S_{i,n} = I + 2T_{i,n}$, where 
$I$ is the identity matrix and $T_{i,n}$ is the $T$-matrix. The method of computing the $T$-matrix is fully discussed in \cite{waterman1969new} and 
\cite{ganesh2008far}.

The following theorem links the single-layer boundary operator with the Krein spectral shift function.
\begin{theorem}\cite{hanisch2020relative}
    Consider $\Omega$ as a domain assembling from individual objects $\Omega_{i}$. Let $V_{k}$ be the single-layer boundary operator defined on the boundary 
    $\partial\Omega = \bigcup_{i = 1}^{N}\partial\Omega_{i}$, and $\tilde{V}_{k}$ is the ``diagonal part'' of $V_{k}$ by restricting the integral 
    kernel to the subset $\bigcup_{i = 1}^{N}\partial\Omega_{i}\times\partial\Omega_{i}\subset\partial\Omega\times\partial\Omega$ then the operator 
    $V_{k}\tilde{V}_{k}^{-1}$ is trace-class and 
    \begin{align*}
        \Xi(k) = \log\det\left(V_{k}\tilde{V}_{k}^{-1}\right).
    \end{align*}

    Then for $k > 0$, we have 
    \begin{align*}
        -\frac{1}{\pi}\emph{\text{Im}}\Xi(k) = \frac{\mathrm{i}}{2\pi}(\Xi(k) - \Xi(-k)) = \xi(k)
    \end{align*}
    and by choosing $h(x) = x$ in \eqref{B-K formula}, this gives the formula 
    \begin{align}\label{slp and matrix}
        \emph{\text{Tr}}\left(\Delta^{\frac{1}{2}} + (N - 1)\Delta_{0}^{\frac{1}{2}} - \sum_{i = 1}^{N}\Delta_{j}^{\frac{1}{2}}\right)  = 
        \int_{0}^{\infty}\xi(k)dk = -\frac{1}{\pi}\int_{0}^{\infty}\Xi(\mathrm{i}k)dk.
    \end{align}
\end{theorem}

Equation \eqref{slp and matrix} is used to compute the Casimir energy between the objects and the formula is written as
\begin{align}\label{KSSF and CasE}
    \mathcal{E} = \frac{\hbar c}{2}\int_{0}^{\infty}\xi(k)dk = -\frac{\hbar c}{2\pi}\int_{0}^{\infty}\Xi(\mathrm{i}k)dk
\end{align}

\begin{remark}
    Note that the integral $\frac{\hbar c}{2}\int_{0}^{\infty}\xi(k)dk$ in \eqref{KSSF and CasE} is not Lebesgue convergent and requires regularisation for its numerical evaluation. The right-hand side integral does not suffer from this issue.
\end{remark}

{\color{red} Corrected to here}


\subsection{Galerkin discretization and boundary element spaces}
In order to compute the integral \eqref{KSSF and CasE}, we need to compute the log determinant of the operators $V_{k}\tilde{V}_{k}^{-1}$. In this section we discuss Galerkin discretisations to compute this quantity.

Define the 
triangulation $\mathcal{T}_{h}$ of the boundary surface $\Gamma$ with triangular surface elements $\tau_{l}$ and associated nodes $\boldsymbol{x}_{i}$ 
s.t. $\overline{\mathcal{T}_{h}} = \bigcup_{l}\overline{\tau_{l}}$, where $h$ is the mesh size and define the space of the continuous piecewise linear functions
\begin{align*}
    P_{h}^{1}(\Gamma) = \{v_{h}\in C^{0}(\Gamma): v_{h}|_{\tau_{l}}\in\mathbb{P}_{1}(\tau_{l}), \ \text{for} \ \ \tau_{l}\in\mathcal{T}_{h}\},
\end{align*}
where $\mathbb{P}_{1}(\tau_{l})$ denotes the space of polynomials of order less than or equal to 1 on $\tau_{1}$
and typically, we use the following $P_{h}^{1}(\Gamma)$ space to discretize $H^{-\frac{1}{2}}(\Gamma)$:
\begin{align*}
    P_{h}^{1}(\Gamma) := \text{span}\{\phi_{j}\} \subset H^{-\frac{1}{2}}(\Gamma)
\end{align*}
with 
\begin{align*}
    \phi_{j}(\boldsymbol{x}_{i}) = \begin{cases}
        1, & i = j,\\
        0, & i\neq j.
    \end{cases}
\end{align*}

Having defined the basis function $P_{h}^{1}(\Gamma)$, we can represent each element inside the Galerkin discretization form. Assume there are $N$ objects,
then the matrix of the operator $V_{k}$ is an $N$ by $N$ block matrix, written as 
\begin{align}\label{matrix V}
    \mathsf{V}_{k} = \mathsf{V}(k) = \begin{bmatrix}
        \mathsf{V}_{11}(k) & \mathsf{V}_{12}(k) & \cdots & \mathsf{V}_{1N}(k) \\
        \mathsf{V}_{21}(k) & \mathsf{V}_{22}(k) & \cdots & \mathsf{V}_{2N}(k) \\
        \vdots & \vdots & \ddots & \vdots \\
        \mathsf{V}_{N1}(k) & \mathsf{V}_{N2}(k) & \cdots & \mathsf{V}_{NN}(k) \\
\end{bmatrix}
\end{align}
and the matrix $\tilde{V}_{k}$ is the diagonal part of $V_{k}$:
\begin{align}\label{matrix tilde V}
    \tilde{\mathsf{V}}_{k} =  \tilde{\mathsf{V}}(k) = \begin{bmatrix}
        \mathsf{V}_{11}(k) & 0      & \cdots & 0 \\
    0      & \mathsf{V}_{22}(k) & \cdots & 0\\
    \vdots & \vdots & \ddots & \vdots \\
    0      & 0      & \cdots & \mathsf{V}_{NN}(k) \\
\end{bmatrix}.
\end{align}
Therefore, the element in the $m$th row and $n$th column of the block matrix $\mathsf{V}_{ij}(k)$ is 
\begin{align}\label{Elements in matrix V}
    \mathsf{V}_{ij}^{(m,n)} (k) = \langle V_{ij}(k)\phi_{m}^{(i)}, \phi_{n}^{(j)}\rangle = 
    \int_{\Gamma}\overline{\phi_{n}^{(j)}}(\boldsymbol{x})\int_{\Gamma}g_{k}(\boldsymbol{x}, \boldsymbol{y})\phi_{m}^{(i)}(\boldsymbol{y})dS_{\boldsymbol{y}}dS_{\boldsymbol{x}},
\end{align}
where $\boldsymbol{\phi}^{(i)} = \begin{bmatrix}
    \phi_{1}^{(i)} & \phi_{2}^{(i)} & \dots & \phi_{N}^{(i)}
\end{bmatrix}$ is the set of basis functions defined on the $i$th object and 
\begin{align*}
    \langle f, g \rangle = \int_{\Gamma}\overline{f(\boldsymbol{x})}g(\boldsymbol{x})dS_{x}
\end{align*}
denotes the standard $L^{2}(\Gamma)$ inner product.


By \eqref{Elements in matrix V}, the explicit form of each element in matrix $\mathsf{V}_{k}$ and $\tilde{\mathsf{V}}_{k}$ is known, therefore,
the value of $\Xi(\mathrm{i}k)$ = $\log\det(V_{\mathrm{i}k}\tilde{V}_{\mathrm{i}k}^{-1})$ can be evaluated by computing 
$\log\frac{\det\mathsf{V}_{\mathrm{i}k}}{\det\tilde{\mathsf{V}}_{\mathrm{i}k}}$ with 
different values of $k$, which is the integrand of the Casimir formula \eqref{KSSF and CasE}. However, by plotting the value of this integrand 
with respect to different values of $\mathrm{i}k$, we found that this function is exponentially decays with increasing the imaginary wavenumber 
$\mathrm{i}k$ (see Figure \ref{The integrand decays exponentially} (Left)) and shares the same trend with $e^{-2Zk}$,  where $Z$ is the minimal distance 
between two objects. This result is proved and summarized in the following theorem. 


\begin{theorem}
    Let $\mathsf{V}_{\mathrm{i}k}$ and $\tilde{\mathsf{V}}_{\mathrm{i}k}$ be the positive definite block matrices defined in \eqref{matrix V} and \eqref{matrix tilde V} 
    and they are partitioned as 
    2 by 2 block matrices
    \begin{align*}
        \mathsf{V}_{\mathrm{i}k} = \begin{bmatrix}
            \mathbb{V}_{11}(\mathrm{i}k) & \mathbb{V}_{12}(\mathrm{i}k)\\
            \mathbb{V}_{21}(\mathrm{i}k) & \mathbb{V}_{22}(\mathrm{i}k)
        \end{bmatrix} \ \ \text{and} \ \   \tilde{\mathsf{V}}_{\mathrm{i}k} = \begin{bmatrix}
            \mathbb{V}_{11}(\mathrm{i}k) & 0\\
            0 & \mathbb{V}_{22}(\mathrm{i}k)
        \end{bmatrix}.
    \end{align*} Denote 
    $\{\lambda_{i}\}_{i}$ and $\{\tilde{\lambda}_{i}\}_{i}$ as the eigenvalues of them, separately and $Z$ be the minimal distance between the objects.
    Then, \begin{align*}
        ||\mathsf{V}_{\mathrm{i}k} - \tilde{\mathsf{V}}_{\mathrm{i}k}||_{2} = \mathcal{O}(e^{-Zk})
    \end{align*}
    and the integrand in \eqref{KSSF and CasE} satisfies
    \begin{align}\label{logdet trend}
        \log\frac{\det\mathsf{V}_{\mathrm{i}k}}{\det\tilde{\mathsf{V}}_{\mathrm{i}k}} = \frac{\mathcal{O}(e^{-2Zk})}{\tilde{\lambda}_{\emph{min}}\emph{gap}_{\emph{min}}},
    \end{align}
    where $\tilde{\lambda}_{\emph{min}} = \emph{min}_{i}\tilde{\lambda}_{i}$ and $\emph{gap}_{\emph{min}} = \emph{min}_{i}\emph{gap}_{i}$ with $\emph{gap}_{i}$ defined as 
    \begin{align*}
        \emph{gap}_{i}:= \begin{cases}
            \emph{min}_{\tilde{\lambda}_{j}\in\mathbb{V}_{22}}|\tilde{\lambda_{i}} - \tilde{\lambda_{j}}|, \ \ \emph{if}\ \ \tilde{\lambda}_{i} \in \lambda(\mathbb{V}_{11})\\
            \emph{min}_{\tilde{\lambda}_{j}\in\mathbb{V}_{11}}|\tilde{\lambda_{i}} - \tilde{\lambda_{j}}|, \ \ \emph{if}\ \  \tilde{\lambda}_{i} \in \lambda(\mathbb{V}_{22}).
        \end{cases}
    \end{align*} 
\end{theorem}
\begin{proof}
    By setting $E_{\mathrm{i}k} = \mathsf{V}_{\mathrm{i}k} - \tilde{\mathsf{V}}_{\mathrm{i}k}$, we have 
    \begin{align*}
        ||E_{\mathrm{i}k}||_{2} = \left|\left|\begin{bmatrix}
            0 & \mathbb{V}_{12}(\mathrm{i}k)\\
            \mathbb{V}_{21}(\mathrm{i}k) & 0
        \end{bmatrix}\right|\right|_{2}.
    \end{align*}
    Since the elements in $\mathbb{V}_{12}(\mathrm{i}k)$ and $\mathbb{V}_{21}(\mathrm{i}k)$ are constructed by \eqref{Elements in matrix V} which includes the 
    Green's function $g_{\mathrm{i}k}$ \eqref{Green's function}, we can conclude $||E_{\mathrm{i}k}||_{2} = \mathcal{O}(e^{-Zk})$, where
    $Z$ is the minimal distance between the objects. 

    To prove \eqref{logdet trend}, we firstly assume the eigenvalues of the matrices $\mathsf{V}_{\mathrm{i}k} $ and $\tilde{\mathsf{V}}_{\mathrm{i}k}$
    are  $\{\lambda_{i}\}_{i}$ and $\{\tilde{\lambda}_{i}\}_{i}$, respectively. Then, according to \cite[Theorem 1]{nakatsukasa2015off}, since $E_{\mathrm{i}k}$ and $\tilde{\mathsf{V}}_{\mathrm{i}k}$
    are symmetric matrices, we have the eigenvalue perturbation bound
    \begin{align}\label{gap}
        |\lambda_{i} - \tilde{\lambda}_{i}| \leq \frac{||E_{\mathrm{i}k}||_{2}^{2}}{\text{gap}_{i}},
    \end{align}
    where 
    \begin{align*}
        \text{gap}_{i}:= \begin{cases}
            \text{min}_{\tilde{\lambda}_{j}\in\mathbb{V}_{22}}|\tilde{\lambda_{i}} - \tilde{\lambda_{j}}|, \ \ \text{if}\ \ \tilde{\lambda}_{i} \in \lambda(\mathbb{V}_{11})\\
            \text{min}_{\tilde{\lambda}_{j}\in\mathbb{V}_{11}}|\tilde{\lambda_{i}} - \tilde{\lambda_{j}}|, \ \ \text{if}\ \  \tilde{\lambda}_{i} \in \lambda(\mathbb{V}_{22}).
        \end{cases}
    \end{align*} 
    Since  $||E_{\mathrm{i}k}||_{2} = \mathcal{O}(e^{-Zk})$, \eqref{gap} becomes $\lambda_{i} - \tilde{\lambda}_{i} = \frac{\mathcal{O}(e^{-2Zk})}{\text{gap}_{i}}$.
    By setting $\tilde{\lambda}_{\text{min}} = \text{min}_{i}\tilde{\lambda}_{i}$ and $\text{gap}_{\text{min}} = \text{min}_{i}\text{gap}_{i}$, we have 
    \begin{align*}
        \log\frac{\det\mathsf{V}_{\mathrm{i}k}}{\det\tilde{\mathsf{V}}_{\mathrm{i}k}} &= \sum_{i}\log\frac{\lambda_{i}}{\tilde{\lambda}_{i}}\\
        &= \sum_{i}\log\left[1+\frac{\mathcal{O}(e^{-2Zk})}{\tilde{\lambda}_{i}\text{gap}_{i}}\right]\\
        &= \sum_{i} \frac{\mathcal{O}(e^{-2Zk})}{\tilde{\lambda}_{i}\text{gap}_{i}} + \text{h.o.t}\\
        &\leq \sum_{i}\frac{\mathcal{O}(e^{-2Zk})}{\tilde{\lambda}_{\text{min}}\text{gap}_{\text{min}}} + \text{h.o.t}\\
        &= \frac{\mathcal{O}(e^{-2Zk})}{\tilde{\lambda}_{\text{min}}\text{gap}_{\text{min}}} 
    \end{align*}
\end{proof}
Note that the numerical experiments indicate that the eigenvalues $\{\tilde{\lambda}_{i}\}_{i}$ and the eigenvalue gaps $\{\text{gap}_{i}\}_{i}$ do not exponentially
depend on $k$.


\begin{figure}[H]
    \centering
    \hspace*{-1cm}\includegraphics[scale = 0.4]{figures/integ_exp_decay.pdf}
    \caption{(Left) The integrand of the Casimir energy whose value exponentially decays with increasing imaginary wavenumber $\mathrm{i}k$. The integrand function 
    shares the same trend with $e^{-2Zk}$, where $Z$ is the minimal distance between two objects. The scatterers are two spheres with equal radii 
    $R = 1$ with minimal distance $Z = 1.5$. (Right) The integrand of the Casimir energy after changing the 
    variable for applying the trapezoid quadrature rule.}
    \label{The integrand decays exponentially}
\end{figure}


Finally, one can apply the normal trapezoidal rule to calculate the integral 
$\int_{0}^{\infty}\Xi(\mathrm{i}k)dk = \int_{0}^{\infty}\log\frac{\det\mathsf{V}_{\mathrm{i}k}}{\det\tilde{V}_{\mathrm{i}k}}dk$ with variable changed. 
The steps are sketched as follows.

\begin{itemize}
    \item Set $f(k) = \log\frac{\det\mathsf{V}_{\mathrm{i}k}}{\det\tilde{\mathsf{V}}_{\mathrm{i}k}}$ and the range of $k$ is from 0 to $\infty$.
    \item Let $k = -\log(y)$, then the integral $\int_{0}^{\infty}\log\frac{\det\mathsf{V}_{\mathrm{i}k}}{\det\tilde{V}_{\mathrm{i}k}}dk$ becomes 
    $\int_{0}^{\infty}f(k)dk = \int_{0}^{1}\frac{f(-\text{log}(y))}{y}dy$.
    \item Set the range of $k$ as $(\text{lb}, \text{ub})$ \footnote{``ub'' is short for upperbound and ``lb'' is short for lowerbound.} and the corresponding 
    range for $y$ is $(e^{-\text{ub}}, e^{-\text{lb}})\subset[0,1]$.
    \item Choose $m$ quadrature points from the interval $(e^{-\text{ub}}, e^{-\text{lb}})$ and use the trapezoidal rule to evaluate the integral 
    $\int_{e^{-\text{ub}}}^{e^{-\text{lb}}}\frac{f(-\text{log}(y))}{y}dy$. Figure \ref{The integrand decays exponentially} (Right) plots the integrand with regard to new 
    variable $y$($= e^{-k}$).

\end{itemize}



 

