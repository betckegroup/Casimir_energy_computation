We have introduced the representation of the Casimir energy in terms of the boundary integral operators and its connection to the relative Krein spectral shift 
function. With the rigorous proof of the Casimir energy formula, we present several numerical methods for 
the calculation of the Casimir energy by applying the spectral properties of the Galerkin discretized form of the boundary 
integral operators. These methods are based on the classical Krylov subspace projection methods and subspaces recycled also significantly reduce the number of the matrix-vector 
multiplications and speed up the calculations for large-scale practical problems.  For the future work, we will consider the calculation of the Casimir energy 
for the electromagnetic scattering case and implement the fast multipole method for larger practical problems.
% and apply the fast multipole method (FMM) 
% when assembling the boundary operator with a shorter assembly time but slower matvecs and compare with the method without implementing FMM.