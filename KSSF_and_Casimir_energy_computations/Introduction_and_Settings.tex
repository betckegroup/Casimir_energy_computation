Casimir interactions are forces between objects such as perfect conductors. Hendrik Casimir predicted and computed this effect in the special case of two planar 
conductors in 1948 using a divergent formula for the zero point energy and applying regularisation to it \cite{casimir1948attraction}. This resulted in the 
famous formula for the attractive Casimir force per unit area 
\begin{align*}
    F(a) = -\frac{1}{A}\frac{\partial \mathcal{E}}{\partial a} = -\frac{\hbar c\pi^{2}}{240a^{4}},
\end{align*}
between two perfectly conducting plates, where $A$ is the cross-sectional area of the boundary plates and $\mathcal{E}$ is the Casimir energy as 
computed from a zeta regularised mode sum. The result here is for the electromagnetic field which differs by a factor of two from the force resulting from a 
massless scalar field.
This force was measured experimentally by Sparnaay 
about 10 years later \cite{sparnaay1958measurements} and the Casimir effect has since become famous for its intriguing derivation and its counterintuitive nature.
In 1996, precision measurements of the Casimir force between 
 extended bodies were conducted by S.K. Lamoreaux \cite{lamoreaux1997demonstration}  confirming the theoretical predictions including corrections for 
realistic materials. From 2000 to 2008, the Casimir force has been measured in various 
experimental configurations, such as cylinder-cylinder \cite{ederth2000template}, plate-plate \cite{bressi2002measurement}, 
sphere-plate \cite{krause2007experimental} and sphere-comb \cite{chan2008measurement}. 
The presence of the Casimir force has also been quoted as evidence for the zero point energy of the vacuum having direct physical significance.
The classical way to compute Casimir forces mimicks Casimir's original computation and is based on zeta function regularisation of the vacuum energy. 
This has been carried out for a number of particular geometric situations (see \cite{bordag2001new, bordag2009advances, elizalde1989expressions, elizalde1990heat, kirsten2001spectral} and references therein). 
The derivations are usually based on special functions and their properties and require explicit knowledge of the spectrum of the Laplace operator.

In 1960s, Lifshitz and collaborators extended and modified this theory to the case of dielectric media \cite{dzyaloshinskii1961general} 
gave derivations based on the stress energy tensor. It has also been realised by quantum field theorists  
(see e.g. \cite{brown1969vacuum, dzyaloshinskii1961general, deutsch1979boundary, kay1979casimir, scharf1992casimir}) with various degrees of mathematical rigour that 
the stress energy approach yields Casimir's formula directly without the need for renormalisation or artificial regularisation.
This tensor is defined by comparing the induced vacuum states of the quantum field with boundary conditions and the free theory. 
Once the renormalised stress energy tensor is mathematically defined, the computation of the Casimir energy density becomes a problem of spectral 
geometry (see e.g. \cite{fulling2007vacuum}). The renormalised stress energy tensor and its relation to the Casimir effect can be understood at the 
level of rigour of axiomatic algebraic quantum field theory. We note however that the computation of the local energy density is non-local and requires 
some knowledge of the spectral resolution of the Laplace operator, the corresponding problem of numerical analysis is therefore extremely hard.

Lifshitz and collaborators also offered an alternative description based on the van der Waals forces between molecules.
The plates consist of a collection of atomic-scale electric dipoles randomly oriented in the absence of the external forcing field. Quantum and thermal 
fluctuations may make the dipoles align spontaneously, resulting in a net electric dipole moment. The dipoles in the opposite plate feel this field
across the gap and align as well. The two net electric dipole moments make the two plates attract each other. This approach emphasizes the 
influence from the materials more than the fluctuations in the empty space between the plates.

Somewhat independently from the spectral approach determinant formulae based on the van der Waal's mechanism were derived by various authors. 
We note here Renne \cite{renne1971microscopic} who gives a determinant formula for the van der Waals force based on microscopic considerations. Various other authors 
give path-integral considerations based on considerations of surface current fluctuations 
\cite{bimonte2017nonequilibrium, emig2007casimir, emig2006casimir, EGJK2008, emig2008casimir, kenneth2006opposites, kenneth2008casimir, milton2008multiple, rahi2009scattering}. The final formulae proved suitable for numerical schemes 
and were also very useful to obtain asymptotic formulae for Casimir forces for large and small separations. The mathematical relation between the various 
approaches remained unclear, with proofs of equality only available in special cases.
A full mathematical justification of the determinant formulae as the trace of an operator describing the Casimir energy was only recently achieved in 
\cite{MR4484208} for the scalar field and \cite{strohmaier2021classical} for the electromagnetic field. It was also proved recently in \cite{fang2021mathematical} that the 
formulae arising from Zeta regularistation, from the stress energy tensor, and from the determinant of the single layer operator all give the same Casimir forces.

We will now describe the precise mathematical setting and review the theory.
Let $\Omega \subset \mathbb{R}^{d}$ be a non-empty bounded open subset with Lipschitz boundary $\partial \Omega$, which is the union of connected open 
sets $\Omega_{j}$, for $j = 1, \dots, N$. We assume that the complement $\mathbb{R}^{d} \backslash \Omega$ of $\Omega$ is connected and the closures of $\Omega_{j}$ 
are pairwise non-intersecting.
We denote the  $N$ connected components of the boundary $\partial\Omega$ by $\partial\Omega_{j}$. 
We will think of the open set $\Omega$ as a collection of objects $\Omega_{j}$ placed in $\mathbb{R}^{d}$ and will refer to them as {\sl obstacles}.

Then, several unbounded self-adjoint operators densely defined in $L^{2}(\mathbb{R}^{d})$
can be defined.
\begin{itemize}
    \item The operator $\Delta$ is the Laplace operator with Dirichlet boundary conditions on $\partial\Omega$.
    \item For $j = 1, \dots, N$, the operator $\Delta_{j}$ is the Laplace operator with Dirichlet boundary conditions on $\partial\Omega_{j}$.
    \item The operator $\Delta_{0}$ is the ``free'' Laplace operator on $\mathbb{R}^{d}$ with domain $H^{2}(\mathbb{R}^{d})$.
\end{itemize}

These operators contain the dense set $C^\infty_0(\mathbb{R}^d \setminus \partial \Omega)$ in their domains.
If $f: \mathbb{R} \to \mathbb{R}$ is a polynomially bounded function this set is also contained in the domain of the operators
$f(\Delta^{\frac{1}{2}}), f(\Delta_{j}^{\frac{1}{2}})$, and $f(\Delta_{0}^{\frac{1}{2}})$, in particular the operator
$$
 D_{f} = f(\Delta^{\frac{1}{2}}) - f(\Delta_{0}^{\frac{1}{2}}) - \left(\sum_{j = 1}^{N}[f(\Delta_{j}^{\frac{1}{2}}) - f(\Delta_{0}^{\frac{1}{2}})]\right)
$$
is densely defined. It was shown in \cite{MR4484208} that under additional analyticity assumptions on $f$ the operator
$D_{f}$ is bounded and extends by continuity to a trace-class operator on $L^2(\mathbb{R}^{d})$. 
These analyticity assumptions are in particular satisfied by $f(k) = (k^{2}+ m^{2})^{\frac{s}{2}}$ for any $s > 0, m \geq 0$ and one has
\begin{align}\label{trace formula in terms of the boundary op}
    \text{Tr}\left( D_{f} \right)  = \frac{s}{\pi} \sin\left(\frac{\pi}{2} s\right) \int_{m}^{\infty} k (k^{2} + m^{2})^{\frac{s}{2}-1}\Xi(\mathrm{i} k) dk,
\end{align}
where the function $\Xi$ is given by
$$
 \Xi(k) = \log \det V_{k} \tilde V_{k}^{-1}
$$
and the operators $V_{k}$ and $\tilde V_{k}$ are certain boundary layer operators that will be defined later. 
It was proved in  \cite{MR4484208} that the above determinant is well-defined in the sense of Fredholm as the operator $V_{k} \tilde V_{k}^{-1}$ near the positive imaginary axis differs
from the identity operator by a trace-class operator on the Sobolev space $H^\frac{1}{2}(\partial \Omega)$.
We remark here that the paper \cite{MR4484208} assumed the boundary to be smooth and the operators $V_k  \tilde V_k^{-1}$ was considered as a 
map on $L^2(\partial \Omega)$. The main result of the paper also holds for Lipschitz boundaries if $L^2(\partial \Omega)$ is replaced by $H^\frac{1}{2}(\partial \Omega)$. This requires minor modifications of the proof but we will not discuss this further here, as we are now focusing on computational aspects.

We also recall that by the Birman-Krein formula we have for any even function $h \in \mathcal{S}(\mathbb{R})$ the equality
\begin{align}\label{B-K formula}
    \text{Tr}\left(h(\Delta^{\frac{1}{2}}) - h(\Delta_{0}^{\frac{1}{2}}) - \left(\sum_{j = 1}^{N}[h(\Delta_{j}^{\frac{1}{2}}) - h(\Delta_{0}^{\frac{1}{2}})]\right)\right)  = \int_{0}^{\infty}h'(k)\xi(k)dk,
\end{align}
where 
\begin{align*}
    \xi(k) = \frac{1}{2\pi \mathrm{i}}\log\left(\frac{\det(S(k))}{\det(S_{1,k})\cdots\det(S_{N,k}(k))}\right)
\end{align*}
will be called the relative Krein spectral shift function. Here, $S_{j,k}$ are the scattering matrices of $\Delta_{j}$ associated to the objects $\Omega_{j}$. Note here that the class of functions for which this is true can be relaxed to a certain extent, but even the most general version does not allow unbounded functions such as $f(k)$ with $s>0, m\geq 0$.
The relative spectral shift function can however be related via a Laplace transform to the Fourier transform of the relative spectral shift function (see \cite{MR4396069}). Under mild convexity assumptions this can be connected to the Duistermaat-Guillemin trace formula in obstacle scattering theory to give an asymptotic expansion of  $\Xi(k)$ 
in terms of the minimal distance $\delta>0$ between the obstacles and the linearised Poincar\'e map of the bouncing ball orbits between the obstacles of that length. One has
$$
 \Xi(k) =- \sum_{j} \frac{1}{|\det(I - P_{\gamma_j})|^{\frac{1}{2}}} e^{2 i \delta k} + o(e^{- 2 \delta \text{Im}{k}}),
$$
where the sum is over  bouncing ball modes of length $2 \delta$ and $P_{\gamma_j}$ is the associated Poincar\'e map, where $\gamma_{j}$ is the shortest bouncing ball orbits. 

The Casimir energy of the configuration $\Omega$ for a massless scalar field would then be given by $D_f$ in the case when $f(k)=k$ and is therefore equal to
$$
\zeta = \frac{\hbar c}{2 \pi} \int _{0}^{\infty} \Xi(\mathrm{i}k) dk.
$$

In this paper, we are going to introduce the numerical framework of computing the Casimir energy based on the evaluation of the log determinant of the integral 
operators in the acoustic case \footnote{The mathematical theories and numerical experiments in the Maxwell case have been done as well and they will be 
reported in another paper.} in Section \ref{Numerical methods for computing the Casimir energy}. Afterwards,
two efficient methods for computing the integrand of the Casimir energy will be illustrated in Section \ref{Krylov subspace for generalized eigenvalue problem}
which makes us compute the large-scale problem efficiently. In Section \ref{Numerical experiments}, several examples on computing the Casimir energy between 
 compact objects will be shown and we will also compare our results with others computed in other methods. Note that all the tests and examples in this paper were computed 
with version 0.2.4 of the Bempp-cl library \cite{scroggs2017software}. Finally, Section \ref{Conclusion} will conclude 
our paper and discuss the future plan as well.

% and discuss the spectral properties 
% of the block matrices constructed from the integral operators in Section \ref{Spectral property of the integral operators}.