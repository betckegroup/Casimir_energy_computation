Since late 1940s, the advanced understanding of vacuum sector of quantum electrodynamics has been developed and one of the most seminal predictions is
the vacuum effects (quantum fluctuations of the electromagnetic field) can induce attractive forces between two uncharged, perfectly conducting, parallel 
plates. This phenomenon is called \emph{Casimir effect}, which is firstly proposed by H.B.G. Casimir in 1940s \cite{casimir1948attraction}. To obtain the formula 
of the Casimir energy, he considered the space between these two plates as a type of electromagnetic cavity, solved the classical Maxwell equations with 
certain boundary condition for all the valid excitations of the electromagnetic field in this cavity and obtained countably infinite electromagnetic mode 
frequencies $\{\omega_{n}(a)\}$, where $a$ is the surface-surface distance. Accordingly, the electromagnetic zero-point energy of the $n$th cavity mode is 
$\frac{\hbar\omega_{n}(a)}{2}$ and by summing up all the contributions from the modes, he finally get the expression of the \emph{Casimir energy}:
\begin{align*}
    \mathcal{E}(a) = \frac{1}{2}\sum_{n}\hbar\omega_{n}(a),
\end{align*}
which is divergent. However, Casimir derived a finite result for the derivative of this infinite energy fluctuations, 
which is the \emph{Casimir force per unit area}:
\begin{align*}
    F(a) = -\frac{1}{A}\frac{\partial\mathcal{E}}{\partial a} = -\frac{\hbar c\pi^{2}}{240a^{4}},
\end{align*}
where $A$ is the cross-sectional area of the boundary plates. 

In 1960s, Lifshitz generalized this theory to the case of dielectric media \cite{dzyaloshinskii1961general}, which means the boundary surfaces are not perfect conductors but the real-world
materials such us the solid consisting of the atomic constituents. He proposed that any microscopic volume $\Delta V$ of the plates contains a collection of 
atomic-scale electric dipoles without uniform orientation due to the absence of the external forcing field. Once in a while, the quantum and thermal 
fluctuations may make the dipoles align spontaneously, resulting in a net electric dipole moment. It is like van der Waals interaction between the 
atoms and this dipole moment induces a net dipole field. Meanwhile, the dipoles in the opposite plate feel this field
across the gap and align as well. Now, there are two net electric dipole moments which make two plates attract with each other. Lifshitz emphasized the 
influence from the materials more than the fluctuations in the empty space between the plates and this interpretation provides little advantage on the 
computation of the Casimir energy since it is unknown on how to compute each contribution from the volume $\Delta V$. 

Afterwards, there is a decades absence of experimental input on the Casimir effect and finally in 1996, the precise measurements of the Casimir force between 
the extended bodies have been done by S.K. Lamoreaux \cite{lamoreaux1997demonstration}. From 2000 to 2008, the Casimir force has been measured in various 
experimental configurations, such as cylinder-cylinder \cite{ederth2000template}, plate-plate \cite{bressi2002measurement}, 
sphere-plate \cite{krause2007experimental} and sphere-comb \cite{chan2008measurement}. The rapid growth in experimental investigation is followed by the 
theoretical development. From 2007 to 2008, the asymptotic series of the Casimir energy have been explicitly computed by T. Emig's team in both scalar
\cite{emig2008casimir} and vector \cite{emig2007casimir} cases. In 2009, Johnson's team put forward a method of computing the Casimir interactions between 
arbitrary three-dimensional objects with arbitrary material properties \cite{reid2009efficient}, in which the Casimir energy between the perfectly
conducting compact objects can be described as:

\begin{align*}
    \mathcal{E} = -\frac{\hbar c}{2\pi}\int_{0}^{\infty}dk\log\frac{\mathcal{Z}(k)}{\mathcal{Z}_{\infty}(k)},
\end{align*}
where 
\begin{align}\label{functional integration}
    \mathcal{Z}(k) = \int \mathcal{D}\boldsymbol{J}(\boldsymbol{x})e^{\frac{1}{2}\int\int d\boldsymbol{x}d\boldsymbol{y}\boldsymbol{J}(\boldsymbol{x}) \cdot \boldsymbol{G}_{k}(\boldsymbol{x}, \boldsymbol{y}) \cdot \boldsymbol{J}(\boldsymbol{y})}
\end{align}
is a functional integration extending over all possible surface current distributions $\boldsymbol{J}(\boldsymbol{x})$ on the objects and 
$\mathcal{Z}_{\infty}(k)$ is $\mathcal{Z}(k)$ computed with all the objects removed to infinite separation. Moreover, in \eqref{functional integration},
\begin{align*}
    \boldsymbol{G}_{k}(\boldsymbol{x}, \boldsymbol{y}) = \left[1 + \frac{1}{k^{2}}\nabla_{\boldsymbol{x}}\otimes\nabla_{\boldsymbol{y}}\right]\frac{e^{\mathrm{i}k|\boldsymbol{x} - \boldsymbol{y}|}}{4\pi|\boldsymbol{x} - \boldsymbol{y}|}
\end{align*}
is the dyadic/tensor Green's function and $k$ is the wavenumber. Johnson proposed that one could formally apply contour integral arguments to obtain the integral 
along the imaginary axis on which the integrand is nicely behaved (non-oscillatory and exponentially decaying). In this case, the Casimir energy formula 
is rewritten as 
\begin{align}\label{rotated CasE}
    \mathcal{E} = -\frac{\hbar c}{2\pi}\int_{0}^{\infty}dk\log\frac{\mathcal{Z}(\mathrm{i}k)}{\mathcal{Z}_{\infty}(\mathrm{i}k)},
\end{align}
where 
\begin{align}
    \mathcal{Z}(\mathrm{i}k) = \int \mathcal{D}\boldsymbol{J}(\boldsymbol{x})e^{\frac{1}{2}\int\int d\boldsymbol{x}d\boldsymbol{y}\boldsymbol{J}(\boldsymbol{x}) \cdot \boldsymbol{G}_{\mathrm{i}k}(\boldsymbol{x}, \boldsymbol{y}) \cdot \boldsymbol{J}(\boldsymbol{y})}
\end{align}
and 

\begin{align*}
    \boldsymbol{G}_{\mathrm{i}k}(\boldsymbol{x}, \boldsymbol{y}) = \left[1 - \frac{1}{k^{2}}\nabla_{\boldsymbol{x}}\otimes\nabla_{\boldsymbol{y}}\right]\frac{e^{-k|\boldsymbol{x} - \boldsymbol{y}|}}{4\pi|\boldsymbol{x} - \boldsymbol{y}|}.
\end{align*}

This $ \boldsymbol{G}_{\mathrm{i}k}$ is called the Wick-rotated dyadic/tensor Green's function. One can notice that this Green's function is strongly singular 
due to the second order form ($\nabla_{\boldsymbol{x}}\otimes\nabla_{\boldsymbol{y}}$). 

%In Section \ref{Numerical methods for computing the Casimir energy}, we 
%will use the integration by parts to make the integral \eqref{rotated CasE} become a numerically computable form. Moreover, after some mathematical manipulations, 
%we can obtain the formula of the Casimir energy in terms of the integral operators, which will also be discussed in Section \ref{Numerical methods for computing the Casimir energy}.


However, a more rigorous mathematical 
derivation was recently provided by Hanisch, Strohmaier and Waters \cite{hanisch2020relative} who have shown that the natural and well-defined object is the 
integral along the imaginary axis. They also provided a mathematical framework to connect the integral to families of trace formulas and the Casimir 
energy is one of the examples. To be specific, by assuming that the objects $\Omega$ is assembled from individual objects $\Omega_{j}$, for $j = 1, \dots, N$ and 
$\partial\Omega_{j}$ are the $N$ connected components of the boundary $\partial\Omega$. Then, several self-adjoint operators on $L^{2}(\mathbb{R}^{d})$
can be defined for constructing the \emph{Birman-Krein formula} later:
\begin{itemize}
    \item The operator $\Delta$ is the Laplace operator with Dirichlet boundary conditions on $\partial\Omega$.
    \item For $j = 1, \dots, N$, the operator $\Delta_{j}$ is the Laplace operator with Dirichlet boundary conditions on $\partial\Omega_{j}$.
    \item The operator $\Delta_{0}$ is the `free' Laplace operator on $\mathbb{R}^{d}$ with domain $H^{2}(\mathbb{R}^{d})$.
\end{itemize}

Now, the Birman-Krein formula can be written as 
\begin{align}\label{B-K formula}
    \text{Tr}\left(f(\Delta^{\frac{1}{2}}) - f(\Delta_{0}^{\frac{1}{2}}) - \left(\sum_{j = 1}^{N}[f(\Delta_{j}^{\frac{1}{2}}) - f(\Delta_{0}^{\frac{1}{2}})]\right)\right)  = \int_{0}^{\infty}f'(k)\xi(k)dk,
\end{align}
where 
\begin{align*}
    \xi(k) = \frac{1}{2\pi \mathrm{i}}\log\left(\frac{\det(S(k))}{\det(S_{1,k})\cdots\det(S_{N,k})}\right)
\end{align*}
is called the \emph{Krein spectral shift function}. Here, $\det(S_{j,k})$ are the scattering matrices of $\Delta_{j}$ associated to the objects $\Omega_{j}$.

According to \cite{hanisch2020relative}, by setting $f(x) = x$, \eqref{B-K formula} becomes 
\begin{align*}
    \text{Tr}\left(\Delta^{\frac{1}{2}} + (N - 1)\Delta_{0}^{\frac{1}{2}} - \sum_{j = 1}^{N}\Delta_{j}^{\frac{1}{2}}\right)  = \int_{0}^{\infty}\xi(k)dk.
\end{align*}
This formula can be used to compute the Casimir energy and it can be an alternative efficient method based on the numerical evaluation of scattering matrices
whose dimension depends on the number of the spherical harmonic functions used in the spherical expansion of the incident/scattering waves.

In this paper, we are going to introduce the numerical framework of computing the Casimir energy based on the evaluation of the log determinant of the integral 
operators in both acoustic and electromagnetic cases in Section \ref{Numerical methods for computing the Casimir energy} and discuss the spectral properties 
of the block matrices constructed from the integral operators in Section \ref{Spectral property of the integral operators}. Afterwards, with these properties, 
an inverse-free method of computing the integrand of the Casimir energy will be illustrated in Section \ref{Krylov subspace for generalized eigenvalue problem}
which makes us compute the large-scale problem efficiently. In Section \ref{Numerical experiments}, several examples on computing the Casimir energy between 
the compact objects will be shown and we will also compare our results with others computed in other methods. Finally, Section \ref{Conclusion} will conclude 
our paper and discuss the future plan as well.